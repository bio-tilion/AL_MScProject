% 200-300 words
\begin{abstract}
Unilever's SEAC division is responsible for the safety and environmental impact assessment of Unilever products. To carry out this task in a sustainable and ethical manner, Unilever promotes the use of non-animal testing methods such as High Throughput Transcriptomics (HTTr) assays, which can be used to test the effect of different concentration of chemicals with different exposure time courses. Time can be factored in using different methodologies, but dynamic methods that model time as a continuous variable are not as well validated as other historically more established approaches.

This project focuses on developing a pipeline for the analysis of RNA-seq time series data by revising some of the most used and popular tools. In particular, the focus of this study is DPGP, a clustering tool that uses a Dirichlet process (DP) to determine the number of clusters, combined with a Gaussian process (GP) for modelling expression levels over time, and Bayesian inference to infer these parameters.

Using DPGP, this study was able (1) to reproduce very similar results to the ones published in its original paper, proving that DPGP can produce very consistent outputs, and (2) to assess its performance on simulated datasets with known characteristics and expected outcomes, which lead to inconclusive interpretations because of the assumptions used by the tool for generating those datasets.

Overall, DPGP has proven to be a potentially good addition to the pipeline. However, it needs to be investigated further in comparison to other similar tools and in the context of a real RNA-seq analysis pipeline.
\end{abstract}